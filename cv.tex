\documentclass{resume} % Use the custom resume.cls style
\usepackage[dvipsnames]{xcolor}
\usepackage{hyperref}
\usepackage{enumitem}
\usepackage[backend=biber, style=ieee, sorting=none]{biblatex}
\addbibresource{references.bib}

\usepackage[left=0.75in,top=0.6in,right=0.75in,bottom=0.1in]{geometry} % Document margins
\newcommand{\tab}[1]{\hspace{.2667\textwidth}\rlap{#1}}
\newcommand{\itab}[1]{\hspace{0em}\rlap{#1}}
\name{David Beinhauer} % Your name
\address{
 GitHub: \href{https://github.com/dbeinhauer}{dbeinhauer} \\
 \href{https://www.linkedin.com/in/david-beinhauer-a0221a247/}{LinkedIn}
}
\address{Phone: (+420) 777-186-273  \\ Email: \texttt{david.beinhauer@email.cz}} % Your phone number and email

\definecolor{CarnegieMellonRed}{RGB}{196,18,48}

\renewenvironment{rSection}[1]{
\sectionskip
\textcolor{CarnegieMellonRed}{\MakeUppercase{#1}}
\sectionlineskip
\hrule
\begin{list}{}{
\setlength{\leftmargin}{1.5em}
}
\item[]
}{
\end{list}
}


\begin{document}

\begin{rSection}{Research Interests}
    I am primarily interested in applying computational neuroscience to clinical treatments, 
    particularly neurorehabilitation and neural restoration. My goal is to develop machine 
    learning-driven models for personalized therapies that improve patient outcomes in 
    individuals with neurological impairments. I aim to bridge the gap between computational 
    models and clinical interventions, exploring neural prosthetics and other neurorestorative 
    strategies to restore brain function and enhance the quality of life.
    My work is driven by the goal of translating cutting-edge neuroscience and 
    AI techniques into real-world solutions for medical challenges.
\end{rSection}


\begin{rSection}{Research Experience}
\begin{rProject}{\large Modeling spatio-temporal dynamics in primary visual cortex 
using deep neural network model}{}
{Supervisor: \href{https://www.mff.cuni.cz/en/faculty/organizational-structure/people?hdl=11457}
{Dr. Ján Antolík}}
{CSNG}
    \item \textbf{Master Thesis} \vline \: 
    \href{https://github.com/dbeinhauer/mcs-source}{GitHub (Code)},
    \href{https://github.com/dbeinhauer/mcs-thesis}{GitHub (Thesis)}
    \item \textbf{Advancing Visual Neuroscience with Biologically Inspired Recurrent Neural Networks} - Poster presentation at the  \href{https://www.neuromonster.org/poster-session-1}{\emph{Sixth International Conference on the Mathematics of Neuroscience and AI}}, May 2025. Presented by my supervisor, Dr. Ján Antolík, on my behalf.
    \item Developed a recurrent neural network model to simulate neuronal responses in V1 layers IV and II/III, based on stimuli from the Lateral Geniculate Nucleus (LGN).
    \item Incorporated biological constraints to improve model interpretability and realism.
    \item Utilized a data-driven model of the cat V1 cortex (Antolík et al., \emph{PLOS Computational Biology}, 2024) for training data.
    \item Investigated the spatio-temporal dynamics of neuronal responses through simulation.
\end{rProject}


\begin{rProject}{\large Optimization of the Placement of Electric Vehicle Charging Stations}{}
{Supervisor: \href{https://www.mff.cuni.cz/en/faculty/organizational-structure/people?hdl=4171}
{Dr. Martin Pilát}}
{KTIML}
    \item \textbf{Bachelor's thesis} \vline \: 
    \href{https://github.com/dbeinhauer/bcs-source}{GitHub (Code)},
    \href{https://github.com/dbeinhauer/bcs-thesis}{GitHub (Thesis, \em Czech)}
    \item Designed a traffic simulator to evaluate different machine learning 
    optimization approaches.
    \item Implemented and compared three optimization techniques to improve 
    electric vehicle charging station placement.
    \item Analyzed results and proposed an optimal placement strategy.
\end{rProject}

\begin{rProject}{\large Selected Coursework Research Projects}{}
{}{}{}
    \item Parkinson Disease Beta Oscillations 
    {\href{https://github.com/dbeinhauer/parkinson_disease_project}{(GitHub)}}
    \item Differences Between Acetylcholinesterase Inhibitors
    {\href{https://github.com/dbeinhauer/acetylcholinester_inhibitors/blob/main/TeX/main.pdf}{(Report)}}
    \item Ant Colony Model (\emph{Czech})
    {\href{https://github.com/dbeinhauer/ant_colony_model/tree/main}{(GitHub)}}
    \item Visual Search Experiment 
    {\href{https://raw.githack.com/dbeinhauer/visual_search_experiment/main/visual-search-experiment-report.html}{(Report)}}
    \item Eye-Tracking Data Analysis 
    {\href{https://raw.githack.com/dbeinhauer/etra_challenge/main/etra_challenge_report.html}{(Report)}}
\end{rProject}

\end{rSection}

%--------------------PUBLICATIONS AND PRESENTATIONS-------------------

\begin{rSection}{Publications and Presentations}

\begin{enumerate}[leftmargin=*, label={[\arabic*]}] \itemsep -6pt

\item \textbf{Beinhauer, D.}, Kraus, R., Baroni, L., \& Antolík, J. (2025, September--October). 
\textit{Bridging biology and deep learning: Modeling visual cortex with structured recurrent networks} [Poster presentation, planned]. 
Bernstein Conference 2025, Frankfurt am Main, Germany.

\item \textbf{Beinhauer, D.}, Kraus, R., Baroni, L., \& Antolík, J. (2025, May~27). 
\textit{Advancing visual neuroscience with biologically inspired recurrent neural networks} [Poster presentation by J.~Antolík]. 
In \textit{Proceedings of the Sixth International Conference on the Mathematics of Neuroscience and AI}, Split, Croatia.

\end{enumerate}

\end{rSection}

\newpage

%--------------------EDUCATION------------------- 
\begin{rSection}{Education}
{\bf \large M.Sc. in Bioinformatics, Charles University, Prague, Czech Republic} \hfill {2022--2025} \\ 
GPA: \textbf{1.33} (1--4 \vline \: 1 is best)  \hfill

{\bf \large B.Sc. in Computer Science, Charles University, Prague, Czech Republic} \hfill {2019--2022} \\
Specialization: \textbf{Artificial Intelligence} \\
GPA: \textbf{1.28} (scale: 1--4 \vline \: 1 is best)


\end{rSection}


% -------------------INDUSTRY EXPERIENCE-------------------

\begin{rSection}{Industry Research Experience}
\begin{rProject}{\large NLP Internhip in MSD Czech Republic}{Oct 2023 - Present}
{}{}
    \item Member of the Artificial Intelligence team, specializing in NLP applications
    in the pharmaceutical sector.
    \item Developed Large Language Model (LLM)-based solutions for scientific document
    processing and data retrieval.
    \item Researched and implemented Retrieval-Augmented Generation (RAG) 
    for scientific data retrieval.
    \item Led bi-weekly presentations to communicate research findings and progress.
    \item Effectively communicated technical solutions to both specialists and 
    non-specialists.
    \item Delivered a functional demo of a novel LLM-powered retrieval system 
    for internal use.
\end{rProject}

\end{rSection}


% -------------------PUBLICATIONS-------------------
% \begin{rSection}{Publications} \itemsep -2pt
% \leavevmode\printbibliography[heading=none]
% \end{rSection}

% \newpage
\begin{rSection}{Grants and Achievements} \itemsep -2pt

\begin{rProject}{\large Travel Grant to the Bernstein Conference}{2025}{}{}
    \item Awarded a travel grant as an early-career scientist presenting as first author in the poster session at the Bernstein Conference in September 2025 in Frankfurt am Main.
\end{rProject}

\begin{rProject}{\large Nomination for Dean's Award}{2025}{}{}
    \item Currently in process. Nominated for one of the best final theses of the academic year at the faculty, an award recognizing outstanding research quality and contribution.
\end{rProject}

\begin{rProject}{Scholarship for Academic Excellence}{2021, 2023}{}{}
    \item Awarded twice for ranking among the top 10\% of students based on academic performance during both bachelor's and master's studies.
\end{rProject}

\end{rSection}


\begin{rSection}{Skills} \itemsep -2pt
\begin{rSkills}
Computational Neuroscience & 
\begin{rSkillsList}
    \item Machine Learning Applications in Neuroscience Research
    \item Spiking Data Analysis and Cortical Modeling
    \item Spiking Neural Networks (Familiar with)
\end{rSkillsList} \\
Computer Science & 
\begin{rSkillsList}
    \item Machine learning, statistics, data analysis
    \item Mathematical modeling and software development
    \item Data visualization
\end{rSkillsList} \\
Biology & 
\begin{rSkillsList}
    \item Neurobiology fundamentals
    \item Visual processing mechanisms
    \item Cell biology, genomics, and structural biology
\end{rSkillsList} \\
Technical Skills & 
\begin{rSkillsList}
    \item Python
    \item Pytorch, TensorFlow, Scikit-learn
    \item HPC, Git, AWS, Docker
\end{rSkillsList} \\
Languages & 
\begin{rSkillsList}
    \item English (Fluent)
\end{rSkillsList} \\
\end{rSkills}
\end{rSection}
\end{document}
