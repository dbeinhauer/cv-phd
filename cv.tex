\documentclass{resume} % Use the custom resume.cls style
\usepackage[dvipsnames]{xcolor}
\usepackage{hyperref}
\usepackage{enumitem}
\usepackage[backend=biber, style=ieee, sorting=none]{biblatex}
\addbibresource{references.bib}

\usepackage[left=0.75in,top=0.6in,right=0.75in,bottom=0.1in]{geometry} % Document margins
\newcommand{\tab}[1]{\hspace{.2667\textwidth}\rlap{#1}}
\newcommand{\itab}[1]{\hspace{0em}\rlap{#1}}
\name{David Beinhauer} % Your name
\address{
 GitHub: \href{https://github.com/dbeinhauer}{dbeinhauer} \\
 \href{https://www.linkedin.com/in/david-beinhauer-a0221a247/}{LinkedIn}
}
\address{Phone: (+420) 777-186-273  \\ Email: \texttt{david.beinhauer@email.cz}} % Your phone number and email

\definecolor{CarnegieMellonRed}{RGB}{196,18,48}

\renewenvironment{rSection}[1]{
\sectionskip
\textcolor{CarnegieMellonRed}{\MakeUppercase{#1}}
\sectionlineskip
\hrule
\begin{list}{}{
\setlength{\leftmargin}{1.5em}
}
\item[]
}{
\end{list}
}


\begin{document}

\begin{rSection}{Research Interests}
    Aspiring PhD candidate with an interdisciplinary background in computer science and bioinformatics, aiming to advance the field of computational neuroscience. I am particularly interested in applying machine learning techniques to model neural systems, with a focus on developing technologies for neural restoration through brain-machine interfaces.
\end{rSection}

%--------------------EDUCATION------------------- 
\begin{rSection}{Education}

{\bf \large M.Sc. in Bioinformatics}, Charles University, Prague, Czech Republic \hfill {2022--2025} \\ 
GPA: \textbf{1.33} (1--4 \vline \: 1 is best)  \hfill \\
{\em Relevant Coursework: Computational Neuroscience, Spatio-Temporal Modeling of Biological Systems, Cell Biology}

{\bf \large B.Sc. in Computer Science}, Charles University, Prague, Czech Republic \hfill {2019--2022} \\
Specialization: \textbf{Artificial Intelligence} \\
GPA: \textbf{1.28} (scale: 1--4 \vline \: 1 is best) \\
{\em Relevant Coursework: Machine Learning, Deep Learning, Statistics, Data Analysis}

\end{rSection}

%--------------------RESEARCH EXPERIENCE------------------- 
\begin{rSection}{Research Experience}
\begin{rProject}{\large Modeling Visual Cortex with Biologically Inspired Recurrent Neural Networks}{}{}{}
    \item {\bf Master Thesis:} \href{https://github.com/dbeinhauer/mcs-source}{GitHub (Code)} \,/\, \href{https://github.com/dbeinhauer/mcs-thesis}{GitHub (Thesis)}
    \item Developed a biologically realistic model of the primary visual cortex (V1) using modular, multi-layered recurrent neural networks.
    \item Mapped synthetic neurons on a one-to-one basis with those from a state-of-the-art spiking neural network model of V1.
    \item Demonstrated that incorporating biological realism is essential for capturing the temporal dynamics of the visual system.
    \item Study results were presented by my supervisor at a poster session [2]; updated results are scheduled to be presented by me in October [1].
    \item Mentored a fellow bachelor student on extending the core research.
\end{rProject}

\begin{rProject}{\large Optimization of Electric Vehicle Charging Station Placement}{}{}{}
    \item {\bf Bachelor Thesis:} \href{https://github.com/dbeinhauer/bcs-source}{GitHub (Code)} \,/\, \href{https://github.com/dbeinhauer/bcs-thesis}{GitHub (Thesis, \emph{Czech})}
    \item Developed a traffic simulation framework for analyzing optimal charging station placements.
    \item Designed and compared various artificial intelligence techniques for optimization.
    \item Demonstrated that a solution based on genetic algorithms was the most effective.
\end{rProject}

\begin{rProject}{\large Selected Coursework Research Projects}{}{}{}

\item \textbf{Parkinson's Disease Beta Oscillations} — Simulated a spiking network model to reproduce and analyze beta oscillations modulated by deep brain stimulation. {\href{https://github.com/dbeinhauer/parkinson_disease_project}{(GitHub)}}

\item \textbf{Eye-Tracking Data Analysis} — Analyzed gaze trajectories and pupil size variations across tasks involving visual fixation, search, and exploration. {\href{https://raw.githack.com/dbeinhauer/etra_challenge/main/etra_challenge_report.html}{(Report)}}
 
\item \textbf{Visual Search Experiment} — Designed and analyzed a behavioral study on reaction time and accuracy under varying visual load and target conditions. {\href{https://raw.githack.com/dbeinhauer/visual_search_experiment/main/visual-search-experiment-report.html}{(Report)}}
 
\item \textbf{Acetylcholinesterase Inhibitor Comparison} — Analyzed structural differences in ligand binding of two Alzheimer's drug targets using 3D protein models. {\href{https://github.com/dbeinhauer/acetylcholinester_inhibitors/blob/main/TeX/main.pdf}{(Report)}}
 
\end{rProject}

\end{rSection}


\newpage
%--------------------PUBLICATIONS AND PRESENTATIONS-------------------

\begin{rSection}{Publications and Presentations}

\begin{enumerate}[leftmargin=*, label={[\arabic*]}] \itemsep -6pt

\item \textbf{Beinhauer, D.}, Kraus, R., Baroni, L., \& Antolík, J. (2025, October). 
\textit{Bridging biology and deep learning: Modeling visual cortex with structured recurrent networks} [Poster presentation, planned]. 
Bernstein Conference 2025, Frankfurt am Main, Germany.

\item \textbf{Beinhauer, D.}, Kraus, R., Baroni, L., \& Antolík, J. (2025, May~27). 
\textit{Advancing visual neuroscience with biologically inspired recurrent neural networks} [Poster presentation by J.~Antolík]. 
In \textit{Proceedings of the Sixth International Conference on the Mathematics of Neuroscience and AI}, Split, Croatia.

\end{enumerate}

\end{rSection}


% -------------------INDUSTRY EXPERIENCE-------------------

\begin{rSection}{Industry Research Experience}

\begin{rProject}{\large NLP Internship, MSD Czech Republic}{Oct 2023 -- Present}{}{}
    \item Member of the Artificial Intelligence team, focused on Natural Language Processing (NLP) applications in the pharmaceutical domain.
    \item Designed and implemented Large Language Model (LLM)-based solutions for scientific document processing and data retrieval.
    \item Collaborated with medical writers to integrate LLM techniques into the research and drafting of internal medical documentation.
    \item Contributed to the development of a modular NLP framework using state-of-the-art technologies and best software engineering practices.
    \item Regularly communicated technical solutions to both expert and non-expert stakeholders to promote adoption of cutting-edge NLP methods.
\end{rProject}

\end{rSection}


\begin{rSection}{Awards and Honors} \itemsep -2pt

\begin{rProject}{\large Travel Grant to the Bernstein Conference}{2025}{}{}
    \item Awarded a travel grant as an early-career scientist presenting as first author in the poster session at the Bernstein Conference in September 2025 in Frankfurt am Main.
\end{rProject}

\begin{rProject}{\large Nomination for Dean's Award}{2025}{}{}
    \item Results pending. Nominated for one of the best final theses of the academic year at the faculty, an award recognizing outstanding research quality and contribution.
\end{rProject}

\begin{rProject}{Scholarship for Academic Excellence}{2021, 2023}{}{}
    \item Awarded twice for ranking among the top 10\% of students based on academic performance during both bachelor's and master's studies.
\end{rProject}

\end{rSection}


\begin{rSection}{Skills} \itemsep -2pt

\begin{rProject}{Technical Skills}{}{}{}
    \item Programming Languages: Python, C++
    \item Machine Learning Frameworks: PyTorch, TensorFlow, Scikit-learn
    \item Cloud and Computing Tools: AWS, HPC environments, Docker
    \item Development Tools: Git, Jira
\end{rProject}

\begin{rProject}{Theoretical and Analytical Skills}{}{}{}
    \item Computer Science: Machine Learning (Deep Learning, Natural Language Processing), Statistical Analysis, Data Processing for Biological Systems
    \item Biological Sciences: Cell Biology, Genomics, Structural Biology, Epigenetics, Basic Neurobiology
\end{rProject}

\begin{rProject}{Languages}{}{}{}
    \item English: Professional proficiency
    \item Czech: Native speaker
\end{rProject}

\end{rSection}


\end{document}

